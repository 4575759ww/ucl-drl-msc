\chapter{Models}
\label{sec:chapterlabel3}

This chapter will develop a reinforcement learning agent definition to solve the recommendation problem using deep learning techniques. After describing the main motivation behind this approach, details of the baseline model derived from previous work on the DDPG algorithm is presented, and finally, as the contribution to this MSc. project, a bayesian personalised ranking model based on factorization machines is introduced into the deep learning algorithm to try to enhance the policy architecture and therefore the system recommendations.

\section{Motivation}

Previously, reinforcement learning agents have received little attention to solve the recommendation tasks due to their action domain becomes intractable in current large-scale recommender systems. However, the great performance reached in several control tasks by end-to-end RL agents based on deep neural networks \cite{mnih2015human} have demonstrated the capacity to continuously adapt their behaviour without any human intervention and solve challenging human tasks. More interestingly, a single trained general-purpose deep reinforcement agent (using the same hyperparameters) was able to converge and perform well on several task environments with different action space definitions. Consequently, a new opened research area has been using recent breakthroughs in training deep neural networks to derive RL agents models that can adapt to other kind of existing tasks under both discrete and continuous domains and solve tasks that were previously considered intractable.

Another reason that motivates our approach is the sequential nature of the recommendation process. In general, RS usually work in a sequential manner: once the system suggests items to the user, she can accept one of the recommendations. Then, at the next stage a new list of recommended items is calculated based on the user feedback and then presented again to the user. This sequential nature allows us to reformulate the recommendation process as a sequential optimization process where optimal recommendations can depend either on the previous items purchased, and in the order in which those items were purchased. For example, Zimdars et al. \cite{zimdars2001using} suggested the use of a k-order Markov chain model (with k=3) to represent this behaviour by dividing a sequence of transactions $X_1, . . . ,X_T$ into cases $(X_{t-k}, . . . ,X_{t-1},X_t$) for $t = 1, . . . ,T$. and then, build a model to predict the item at the time step t given the other columns. 

As now deep reinforcement learning approaches can converge to good estimations under tasks considered intractable before, it motivates us to evaluate and improve these models under recommendation tasks, which this thesis project is focused.

\section{Baseline Agent Definition}
\label{sec:baseline}

Therefore, the recommender problem can be considered as a sequence of Markov Decision Processes $\langle \mathcal{S}, \mathcal{A}, \mathcal{R}, \mathcal{P}, \gamma \rangle$, where the action space $\mathcal{A}$ is defined as the set of items to recommend, $\mathcal{S}$ represent the current state or item a user is currently consuming, thus both $\mathcal{S}$ and $\mathcal{A}$ represent items as feature vectors $\mathbf{a}$ and $\mathbf{s} \in \mathbb{R}^n$ respectively. $\mathcal{P}: \mathcal{S} \times \mathcal{A} \times \mathcal{S} \rightarrow \mathbb{R}$ is the transition probability distribution to move from state ${s_t}$ to $s_{t+1}$,  $\mathcal{R}: \mathcal{S} \times \mathcal{A} \rightarrow \mathbb{R}$ is the reward value function $\mathcal{R}$ associated with an item accepted by the user, and $\gamma \in [0, 1]$ is the discount factor for future rewards. 

The reinforcement learning agent is then defined as a model-free approach with the main objective to find the optimal policy $\pi$ that maximizes the expected total future reward $\mathbb{E}[R_1]$ over all episodes $R_t = \sum^{T}_{i=t} \gamma^{i-t}r(s_i, a_i)$. The expectation can be obtained using the Bellman equation (presented in section \ref{sec:model-free}): 
\begin{equation}
Q^{\pi}(\mathbf{s}, \mathbf{a}) = r(\mathbf{s}, \mathbf{a}) + \gamma \sum_{\mathbf{s}'}P(\mathbf{s}'|\mathbf{s}, \mathbf{a}) Q^{\pi}(\mathbf{s}', \pi(\mathbf{s}')
\end{equation}

In order to approximate $\pi$ and $Q$ under large-scale action spaces, standard actor-critic approaches need to define a policy by a parametrized actor function approximator $\pi_{\theta} = \arg\max_{\mathbf{a \in \mathcal{A}}} Q(\mathbf{s}, \mathbf{a}) : \mathcal{S} \rightarrow \mathcal{A}$ that would be able to generalize and that does not scales linearly in relation to the number of actions (turning the task into an intractable problem). Based on recent work presented by Lillicrap et al. \cite{lillicrap2015continuous} and Dulac-Arnold et al. \cite{Dulac-Arnold2015}, it is now possible to define generalizable reinforcement agents using a policy architecture with a sub-linear complexity relative to the action space that avoids the heavy cost of evaluating all actions. This actor-critic framework uses a multi-layer neural network as function approximators and is trained using Deep Deterministic Policy Gradient (DDPG).

The action generation function is based on the \textit{Wolpertinger} policy defined in \cite{Dulac-Arnold2015}. First, an initial function $f_{\theta^{\pi}}(s) = \hat{\mathbf{a}}$ provides a proto-action in $\mathbb{R}^n$ for a given state s. $\hat{\mathbf{a}}$ is then mapped to the discrete action set $\mathcal{A}$ by applying the k-nearest-neighbor\footnote{The size of the generated action set k is task specific, and allows for an explicit trade-off between policy quality and speed} function from equation \ref{eq:knnDDPG}, which returns the k closest actions in $\mathcal{A}$ by computing the $L_2$ distance between them.
\begin{equation}
\label{eq:knnDDPG}
g_k(\hat{\mathbf{a}}) = \arg\min^k_{a \in \mathcal{A}} | \mathbf{a} - \mathbf{\hat{\mathbf{a}}} |_2
\end{equation}

Even if $g_k$ has the same complexity as the \textit{argmax} in the Q action-value function, each step of evaluation of the $L_2$ distance is faster than a full value-function evaluation as the lookup is performed in logarithmic time. Finally, the choice of the actual action to be applied to the environment is set according to the highest-scoring action obtained according $Q_{\theta^Q}$ defined in equation \ref{eq:qDDPG}. As a result, the full Wolpertinger policy makes the algorithm significantly more robust to imperfections in the choice of action representation.
\begin{equation}
\label{eq:qDDPG}
\pi_{\theta}(s) = \arg\max_{a \in g_k \circ f_{\theta^{\pi}}(s) } Q_{\theta^Q}(s, a)
\end{equation}

On the other hand, the DDPG implementation maintains two functions: (1) the actor function $\mu(s|\theta_{\mu})$ that obtains the current policy by deterministically mapping states to specific actions; and (2) the critic function $Q(s,a)$ learned by applying the Bellman equation. It then updates the actor by applying the chain rule to the expected return from a start distribution $J = \mathbb{E}_{r_i, si \sim E, a_i \sim \pi}[R_1]$ and with respect to the actor parameters:
\begin{equation}
\label{}
\begin{aligned}
\nabla_{\theta^{\mu}}J & \approx \mathbb{E}_{s_t \sim \rho^{\beta}} [\nabla_{\theta^{\mu}} Q(s, a | \theta^Q)|_{s=s_t, a=\mu(s_t|\theta^{\mu})} ] \\
& = \mathbb{E}_{s_t \sim \rho^{\beta}} [\nabla_a Q(s, a | \theta^Q)|_{s=s_t, a=\mu(s_t)}  \nabla_{\theta^{\mu}} \mu(s|\theta^{\mu})|_{s=s_t}]
\end{aligned}
\end{equation}

In order to maintain stability in the learning algorithm, a separate target networks $\mu'(s|\theta_{\mu'})$ and $Q'(s, a|\theta_{Q'})$ for the actor and critic are used for calculating the target values, where their weight parameters are updated by using an smooth function update $\theta' \leftarrow \tau\theta + (1 - \tau)\theta'$ with $\tau \ll 1$ to main an exponential moving average of the trained parameters $\theta^{\mu}$ and $\theta^{Q}$.
 
Among the main advantages of using this deep learning approach, DDPG introduces an experience replay buffer of size N that helps to reduce the correlation among the sequence of samples used by the RL agent. The mechanism allows the behaviour distribution to be averaged over many of its previous states, smoothing out learning and avoiding instability or divergence in the parameters.

When executing the DDPG algorithm, the critic is trained from samples generated and stored in the replay buffer by the full policy, while the policy gradient $\nabla_a Q_{\theta^Q}(s,a)$ is taken at the true output $\hat{\mathbf{a}} = f_{\theta^{\pi}}(s)$. The target action in the Q-update is generated by the full policy.

\section{FM Agent Model}

Our implementation attempts to replace the k-nearest neighbour computation in the Wolpertinger policy for a Bayesian Personalized Ranking (BPR) model using Factorization Machines to find the top-K items from a given state to then select the item that yields to the highest Q value in the long term. The deep reinforcement agent will finally learn using the same DDPG architecture as the baseline model.

The Bayesian formulation presented by Rendle et al. in \cite{rendle2009bpr} finds the correct personalized ranking for all items $i \in I$ by maximizing the posterior probability $p(\theta | >_u) \approx p(>_u | \theta) p(\theta)$ where $\theta$ represents the parameter vector of an arbitrary model class (e.g. matrix factorization) and $>_u$ is the desired but latent preference structure for user $u$. In order to guarantee a personalized order, totality, antisymmetry and transitivity properties are fulfilled by defining the individual probability that a user really prefers item $i$ to $j$ as $p(i >_u j | \theta ) := \sigma(\hat{x}_{uij}(\theta))$ where $\sigma$ is the sigmoid function and $\hat{x}$ is an is  an  arbitrary  real-valued  function of the model parameter vector which captures the special relationship between user $u$, item $i$ and item $j$, such as Matrix Factorization of adaptive kNN.

Rendel S. showed in \cite{rendle2012factorization} that factorization models, such as the BPR model for top-K recommendation, can be expressed using Factorization Machines, where vectors $x$ are ordered by the score of $\hat{y}(x)$ and optimization is done over pairs of instance vectors $(x_i, x_j) \in \mathcal{D}$.

\subsection{Proposed Approach}

The new policy architecture called \textit{FM-policy} tries to improve how an agent selects the action with the highest Q value by evaluating the top-k returned items from a given state (i.e. the item returned by the actor function and last item consumed by the user) by a pre-trained Factorization Machine model, rather than obtaining the k-nearest neighbors using the euclidean distance. As a result, the FM-policy maintains the sublinear complexity of the Wolpertinger policy but it introduces an stronger (factorization) model. The FM-policy algorithm is described in Algorithm \ref{alg:fmpolicy}.

\begin{center}
 \begin{algorithm}[t]
    \SetAlgoLined\DontPrintSemicolon
    \caption{FM policy}
    \label{alg:fmpolicy}
    
    State $\mathbf{s}$ previously received from environment.\;
    $\hat{\mathbf{a}} = f_{\theta^{\pi}}(\mathbf{s})$ {Receive proto-action from actor}.\;
    $\mathcal{A}_k = g_k(\hat{\mathbf{a}}, \mathbf{s})$ {Retrieve Top-k recommendation using BPR-FM-model}\;
    $\mathbf{a}_{TOP} = \arg \max_{\mathbf{a}_j \in \mathcal{A}_k} Q_{\theta^{Q}}(\mathbf{s}, \mathbf{a}_j)$\;
%    \If{$\hat{\mathbf{a}} == \mathbf{a}_{TOP}$}{
%	$updateBPRModel(\mathbf{s}, \mathbf{a})$\;
%    }
    Apply $\mathbf{a}_{TOP}$ to environment; receive $r$, $\mathbf{s}'$\;
  \end{algorithm}
%\end{minipage}
\end{center}
After the actor function provides a proto-action $\hat{\mathbf{a}}$ in $\mathbb{R}$ given the current state or last consumed item by the user, a pre-trained BPR-FM model obtains the the Top-K recommendations based on the current observation: $(\hat{\mathbf{a}}, \mathbf{s})$. Then, the item in $\mathcal{A}_k$ that yields to the highest reward is applied to the environment.

Similarly to the baseline model, when executing the DDPG algorithm, the critic is trained from samples generated and stored in the replay buffer by the FM-policy, while the policy gradient is taken at the true output in the actor function $f_{\theta^{\pi}}(s)$. As a reference, the pseudocode of the DDPG algorithm is included in Appendix \ref{app:ddpgfm}.