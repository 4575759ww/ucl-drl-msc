\addcontentsline{toc}{chapter}{Appendices}

% The \appendix command resets the chapter counter, and changes the chapter numbering scheme to capital letters.
%\chapter{Appendices}
\appendix
\chapter{DDPG Algorithm using a Custom Policy Architecture}
\label{app:ddpgfm}
\begin{center}
 \begin{algorithm}[!htbp]
%    \SetAlgoLined
    \DontPrintSemicolon
    \caption{DDPG algorithm}
    \label{alg:ddpg_alg}

    Randomly initialize actor and critic networks $f_{\theta^{\pi}}$ $Q_{\theta^Q}$\;
    Initialize target network $f_{\theta^{\pi}}$ and $Q_{\theta^{Q}}$ with weights
    $\theta^{Q'} \leftarrow \theta^Q$, $\theta^{\pi'} \leftarrow \theta^{\pi}$\;
    Initialize policy model $g$ with elements in $\mathcal{A}$\;
    Initialize replay buffer B\;
    \For{episode = 1 \KwTo $M$}{
    	Initialize a random process $\mathcal{N}$ for action exploration\;
	Receive initial observation state $\mathbf{s}_1$ from environment\;
	\For{t = 1 \KwTo $T$}{
		Select action $\mathbf{a}_t = \pi_{\theta(\mathbf{s}_t)}$ according current policy and exploration method\;
		Send action $\mathbf{a}_t$ and observe reward  $r_t$ and new state $\mathbf{s}_{t + 1}$\;
		Store transition $(\mathbf{s}_t, \mathbf{a}_t, r_t, \,mathbf{s}_{t+1})$ in B\;
		Sample a random minibatch of N transitions from B\;
		Set $y_i = r_i + \gamma \cdot Q_{\theta^{Q'}}(\mathbf{s}_{i+1}, \pi_{\theta'}(\mathbf{s}_{i+1}))$\;
		Update the critic by minimizing the loss: \;
		$L(\theta^Q) = \frac{1}{N}\sum_{i}[y_i - Q_{\theta^{Q}}(\mathbf{s_i}, \mathbf{a}_i)]^2$\;
		Update the actor using the sampled gradient:\;
		$\nabla_{\theta^{\pi}}f_{\theta^{\pi}}|_{\mathbf{s}_i} \approx \frac{1}{N}\sum_i \nabla_{\mathbf{a}}Q_{\theta^{Q}}(\mathbf{s}, \mathbf{\hat{a}}) |_{\mathbf{\hat{a}}=f_{\theta^{\pi}}(\mathbf{s}_i)} \cdot \nabla_{\theta^{\pi}}f_{\theta^{\pi}}(\mathbf{s}) |_{\mathbf{s}_i}$\;
		Update the target networks:\;
		$\theta^{Q'} \leftarrow \tau\theta^Q + (1 - \tau)\theta^{Q'}$\;
		$\theta^{\pi'} \leftarrow \tau\theta^{\pi} + (1 - \tau)\theta^{\pi'}$\;
	}
    }
  \end{algorithm}
\end{center}

The pseudocode presented above is based in the work presented by Dulac-Arnold et al. in \cite{Dulac-Arnold2015}

\chapter{Transition probability Matrix algorithm}
\label{app:trans_prob_alg}

\begin{center}
 \begin{algorithm}[!htbp]
%    \SetAlgoLined
    \DontPrintSemicolon
    \caption{Transition Probability Matrix(R)}
    \label{alg:transition_matrix}
    
    \KwInput{$R$: Initial User Rating Matrix}
    \KwResult{$\mathcal{P}$: Transition Probability Matrix}
    %$\textbf{e} \leftarrow trigger\_candidate(\textbf{x})$\;
    $S \leftarrow ComputeCosineSimilarityMatrix(R)$\;
    \For{$i \leftarrow 1$ \KwTo $m$}{ 
        $sum \leftarrow \sum^{m}_{j=1} S_{ij}$\;
    \For{$j \leftarrow 1$ \KwTo $m$}{
    \tcc{user walks through a direct neighborith probability $\beta$ or jumps to an arbitrary item with uniform probability}
    $\mathcal{P}_{ij} \leftarrow \beta S_{ij} / sum + (1 - \beta) / m$\;
    }
    }
    \KwRet $\mathcal{P}$
  \end{algorithm}
%\end{minipage}
\end{center}

\chapter{Recommender System Simulated Environment}
\label{app:simulated_env}

\begin{center}
 \begin{algorithm}[!htbp]
    \SetAlgoVlined
    \DontPrintSemicolon
    \caption{Simulated environment}
    \label{alg:simulated_env}

    \KwInput{$R$: Ratings Matrix; $\mathcal{P}$: Transition Prob. Matrix; $\mathcal{I}$: Items set\;}
%    \KwResult{$\textbf{s}(\textbf{e}, \textbf{a}; \textbf{x}, \boldsymbol{\lambda})$}
    
    $u_{random} \leftarrow selectRandomUser(R)$ \textit{\# selects an active user}\;
    $state \leftarrow selectLikelyItem(\mathcal{I}, u_{random})$ \textit{\# initial guided exploration}\;
    $term \leftarrow False$\;
    \While{$not$ $term$}{
    	$action \leftarrow waitForAgentAction()$\;
	\tcc{probability of choosing action given current state}
	$\rho_{action} = \leftarrow computeTransition(\mathcal{P}, state, action)$\;
	$\mathit{rating} \leftarrow getUserRating(u_{random}, action)$\;
	\tcc{get the mean and std dev of the transition probability distr. for a given state}
	$\mu, \sigma \leftarrow transProbDistribution(\mathcal{P}, state)$\;
	$\mathit{threshold} \leftarrow \mu$\;
	$is\_chosen \leftarrow \rho_{action} > \mathit{threshold}$ \textbf{and} $\mathit{rating} > 0$\;
	\eIf{$is\_chosen$}{
		$\rho_{end} \leftarrow 0.1$\;
		$reward \leftarrow rating$\;
	}{
		$\rho_{end} \leftarrow 0.2$\;
		$reward \leftarrow 0$\;
		$action \leftarrow selectRandomItem(u_{random})$\;
	}
	$state \leftarrow action$\;
	\If{$random.uniform() < \rho_{end}$}{
		$term \leftarrow True$\;
	}
	$sendSignalToAgent(state, reward, term)$\;
    }
  \end{algorithm}
\end{center}

\chapter{Hyperparameter configuration for models}
\label{app:hyperparameter}

\begin{table}[!htbp]
%\begin{center}
\centering
\begin{tabular}{ |l|c|c|c| }
  \hline
%  \multicolumn{2}{|c|}{100K Movielens} & 1M Movielens \\
  \textbf{Hyperparameter}& \textbf{DRL-kNN-CB} & \textbf{DRL-kNN-CF} & \textbf{DRL-FM} \\
  \hline
  Hidden Layer 1 Size & 400 & 1,000 & 400 \\ \hline
  Hidden Layer 2 Size & 300 & 1,000 & 400 \\ \hline
  Replay Memory size & \multicolumn{3}{c|}{50,000} \\ \hline
  $k-NN/Top-K$ & \multicolumn{3}{c|}{5\% of action set size} \\ \hline
  $\mu$ $L_2$ reg. & \multicolumn{3}{c|}{0} \\ \hline
  $\mathcal{Q}$ $L_2$ reg. & \multicolumn{3}{c|}{0.01} \\ \hline
  $\nabla \mu$ learning rate & \multicolumn{3}{c|}{0.0001} \\ \hline
  $\nabla \mathcal{Q}$ learning rate & \multicolumn{3}{c|}{0.001} \\ \hline
  $\tau$ & \multicolumn{3}{c|}{0.001} \\
  $\gamma$ & \multicolumn{3}{c|}{0.99} \\
  \hline
\end{tabular}
%\end{center}
\caption{Hyperparameter and network settings for DRL models}
\label{table:hyperparameters}
\end{table}

\chapter{Implementation notes and Installation Requirements}
\label{app:implementation}

\section{System Dependencies}

The following packages are needed to execute the project models

\begin{itemize}
\item Python 2.7
\item Virtualenv
\item Tensorflow
\item Jupyter notebook
\item Graphlab Create API 2.1 (with Academic License)
\end{itemize}

\section{Setup}

To setup the environment, the following commands need to be executed:

\begin{lstlisting}[language=bash]
$ git clone https://github.com/santteegt/ucl-drl-msc
$ cd ucl-drl-msc
$ git submodule update --init --recursive
$ mkdir .venv
$ virtualenv --system-site-packages --python=python2.7 .venv/
$ source .venv/bin/activate

(venv)$ cd gym
(venv)$ pip install -e .
# installation for Mac OS X. For other platforms, 
# refer to 
# https://www.tensorflow.org/versions/r0.9/get_started/os_setup.html
(venv)$ export TF_BINARY_URL=
https://storage.googleapis.com/tensorflow/mac/tensorflow-0.9.0-py2-none-any.whl
(venv)$ pip install --upgrade $TF_BINARY_URL
(venv)$ pip install pymongo pandas gensim fastFM matplotlib scikit-learn scipy
(venv)$ pip install --upgrade --no-cache-dir 
https://get.graphlab.com/GraphLab-Create/2.1/hernan.toral.15@ucl.ac.uk/
55E9-2088-3AF8-120F-D171-1AAB-95A3-B077/GraphLab-Create-License.tar.gz
\end{lstlisting}

\section{Datasets}

Movielens datasets files were pre-processed for running the models. They can be downloaded from the following URL: \url{https://mega.nz/#F!oFYFmZyQ!UERNs4e3Jvvkml2zNj2ngA}. Then, you can follow the instructions below:

\begin{itemize}
\item Extract and copy the contents of \textit{FMmodel.zip} to the \textit{data} folder
\item To use the Movielens-100k dataset, extract the contents of the \textit{data-movielens100k.zip} file into the \textit{data} folder
\item To use the Movielens-1M dataset, extract the contents of the \textit{data-movielens1m.zip} file into the 'textit{data} folder
\end{itemize}

\section{Running the environment}

To make the running process more easier, executable files for each experiment are provided under the `bin` folder (make sure to activate the virtualenv before running the exec file):

\begin{enumerate}
\item \textbf{DRL-kNN-CB}: \textit{run-v2.sh} runs the model for the Movielens 100k dataset, while \textit{run-v0.sh} runs the model for the Movielens 1M dataset
\item \textbf{DRL-kNN-CF}: \textit{run-v1.sh} runs the model for the Movielens 100k dataset.
\item \textbf{DRL-FM}: \textit{run-v3.sh} runs the model for the Movielens 100k dataset.
\item \textbf{Random agent}: \textit{run-v1-random.sh} and \textit{run-v2-random.sh} runs the DDPG algorithm using a random agent
\end{enumerate}

Human readable rendering of recommendations made while running the model can be seen using the following command:

\begin{lstlisting}[language=bash]
(venv)$ cd ucl-drl-msc
(venv)$ tail -F run.log
\end{lstlisting}

Reward logs obtained by an agent while running the model can be read using the following command:

\begin{lstlisting}[language=bash]
(venv)$ cd ucl-drl-msc
(venv)$ tail -F ddpg-results/experiment1/output.log
\end{lstlisting}

Additionally, to visualize the model parameters during training, you can run the Tensorflow dashboard using the following command:

\begin{lstlisting}[language=bash]
(venv)$ cd ucl-drl-msc
(venv)$ python src/dashboard.py --exdir ddpg-results/
\end{lstlisting}

Finally, to stop the training process, it is necessary to run the following command:

\begin{lstlisting}[language=bash]
(venv)$ cd ucl-drl-msc/bin
(venv)$ sh stop.sh
\end{lstlisting}

\section{License}

The Project code is mostly developed under the Apache2 license, except for the module developed using Graphlab Create which is based on an Academic License provided by Turi.

%\chapter{Colophon}
%\label{appendixlabel3}
%\textit{This is a description of the tools you used to make your thesis. It helps people make future documents, reminds you, and looks good.}
%
%\textit{(example)} This document was set in the Times Roman typeface using \LaTeX\ and Bib\TeX , composed with a text editor. 
 % description of document, e.g. type faces, TeX used, TeXmaker, packages and things used for figures. Like a computational details section.
% e.g. http://tex.stackexchange.com/questions/63468/what-is-best-way-to-mention-that-a-document-has-been-typeset-with-tex#63503

% Side note:
%http://tex.stackexchange.com/questions/1319/showcase-of-beautiful-typography-done-in-tex-friends